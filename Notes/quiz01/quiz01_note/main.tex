% --- UNIVERSAL PREAMBLE BLOCK ---
\documentclass[11pt, a4paper]{article}
\usepackage[a4paper, top=2.5cm, bottom=2.5cm, left=2cm, right=2cm]{geometry}
\usepackage{fontspec}

\usepackage[english, bidi=basic, provide=*]{babel}
\babelprovide[import, onchar=ids fonts]{english}

% Set default/Latin font to Sans Serif in the main (rm) slot
\babelfont{rm}{Noto Sans}

% Math and layout packages
\usepackage{amsmath}
\usepackage{booktabs}
\usepackage{caption}
\usepackage{pgfplots}
\pgfplotsset{compat=1.18}
\usepackage{tikz}
\usepackage{graphicx}
\usepackage{subcaption}

\title{\textbf{Complete Statistics Assignment Solutions}}
\author{
    \textbf{[Md. Monjurul Hasan Bhuiyan]} \\ 
    \textit{Student ID: [10000*****4} \\
    \textit{Course: [STA201]}
}
\date{\today}

\begin{document}

\maketitle

\vspace{0.5cm}

\tableofcontents
\vspace{1cm}
\hrule
\vspace{1cm}
\newpage


% ==========================================

% SECTION 1.3: [Insert Title Here]

% ==========================================

\section{1.3 and  1.4}


\begin{figure}[htbp]

     \centering

     % First row: Two images side-by-side

     \begin{subfigure}[b]{0.45\textwidth}

         \centering

         \includegraphics[width=\linewidth]{image.png}

         \caption{Caption for Figure 1}

         \label{fig:image1}

     \end{subfigure}

     \hfill

     \begin{subfigure}[b]{0.45\textwidth}

         \centering

         \includegraphics[width=\linewidth]{af.png}

         \caption{Caption for Figure 2}

         \label{fig:image2}

     \end{subfigure}


     \vspace{1cm} % Adds vertical space between rows


     % Second row: Third image centered below

     \begin{subfigure}[b]{0.6\textwidth}

         \centering

         \includegraphics[width=\linewidth]{wgra.png}

         \caption{Caption for Figure 3}

         \label{fig:image3}

     \end{subfigure}

     

     \caption{Combined visualization of quantitative data.}

     \label{fig:combined_plots}

\end{figure}
% ==========================================
% SECTION 1: QUANTITATIVE DATA (CONTINUOUS)
% ==========================================
\section{Quantitative Data: Frequency Distributions}

\subsection{Example 1: Income of Students}
\textbf{Data Provided:} Income in Tk vs. No. of students ($n=52$).

\subsubsection*{a. Data Summarization}
\begin{table}[h]
\centering
\renewcommand{\arraystretch}{1.2}
\begin{tabular}{cccccc}
\toprule
\textbf{Income (Tk)} & \textbf{Frequency ($f$)} & \textbf{Cum. Freq (cf)} & \textbf{Rel. Freq (rf)} & \textbf{rf Percent (\%)} & \textbf{Cum. Percent (\%)} \\
\midrule
50--100   & 8  & 8  & $8/52 \approx 0.154$  & 15.4 & 15.4 \\
100--150  & 11 & 19 & $11/52 \approx 0.212$ & 21.2 & 36.6 \\
150--200  & 10 & 29 & $10/52 \approx 0.192$ & 19.2 & 55.8 \\
200--250  & 20 & 49 & $20/52 \approx 0.385$ & 38.5 & 94.3 \\
250--300  & 3  & 52 & $3/52 \approx 0.058$  & 5.8  & 100.1 \\
\midrule
\textbf{Total} & \textbf{52} & & \textbf{1.001} & \textbf{100.1} & \\
\bottomrule
\end{tabular}
\end{table}
\textit{Note: Total percentage is slightly above 100\% due to standard rounding.}

\subsubsection*{b. Analytical Questions \& Conclusion}
\textbf{3. How many students (\% or proportion) have income less than 150 Tk?}\\
Answer: Students in the 50-100 and 100-150 groups. $8 + 11 = 19$ students. As a percentage: $15.4\% + 21.2\% = 36.6\%$.

\textbf{4. How many students have income equal to or more than 200 Tk?}\\
Answer: Students in the 200-250 and 250-300 groups. $20 + 3 = 23$ students. As a percentage: $38.5\% + 5.8\% = 44.3\%$.

\textbf{Conclusion / Comment:}\\
A maximum of $38.5\%$ of students have an income between 200-250 Tk, while the minimum number of students ($5.8\%$) fall into the highest income bracket of 250-300 Tk.

\subsubsection*{c. Graphs: Histogram and Frequency Polygon}
\begin{center}
\begin{tikzpicture}
\begin{axis}[
    ybar interval, ymin=0, ymax=22, xmin=50, xmax=300,
    xlabel={Income (Tk)}, ylabel={Number of Students ($f$)},
    xtick={50,100,150,200,250,300}, width=0.48\textwidth, height=6cm, fill=cyan!70!black
]
\addplot coordinates {(50, 8) (100, 11) (150, 10) (200, 20) (250, 3) (300, 0)};
\end{axis}
\end{tikzpicture}
\quad
\begin{tikzpicture}
\begin{axis}[
    ymin=0, ymax=22, xmin=0, xmax=350,
    xlabel={Income (Tk) - Midpoints}, ylabel={Frequency ($f$)},
    xtick={25, 75, 125, 175, 225, 275, 325}, width=0.48\textwidth, height=6cm
]
\addplot[mark=*, thick, red] coordinates {(25, 0) (75, 8) (125, 11) (175, 10) (225, 20) (275, 3) (325, 0)};
\end{axis}
\end{tikzpicture}
\end{center}


\vspace{1cm}
% ------------------------------------------
\subsection{Example 2: Age Distribution (15-40 years)}
\textbf{Data Provided:} Age vs. Number of Persons ($n=35$).

\subsubsection*{a. Data Summarization}
\begin{table}[h]
\centering
\begin{tabular}{cccccc}
\toprule
\textbf{Age (years)} & \textbf{Freq ($f$)} & \textbf{Cum. Freq (cf)} & \textbf{Rel. Freq (rf)} & \textbf{rf Percent (\%)} & \textbf{Cum. Percent (\%)} \\
\midrule
15--20 & 9  & 9  & 0.257 & 25.7 & 25.7 \\
20--25 & 7  & 16 & 0.200 & 20.0 & 45.7 \\
25--30 & 12 & 28 & 0.343 & 34.3 & 80.0 \\
30--35 & 5  & 33 & 0.143 & 14.3 & 94.3 \\
35--40 & 2  & 35 & 0.057 & 5.7  & 100.0 \\
\midrule
\textbf{Total} & \textbf{35} & & \textbf{1.000} & \textbf{100.0} & \\
\bottomrule
\end{tabular}
\end{table}

\subsubsection*{b. Graphs and Conclusion}
\textbf{Conclusion:} The largest age demographic in this dataset is 25-30 years, comprising $34.3\%$ of the individuals. The distribution is heavily skewed to the right (younger population), with $80\%$ of individuals being under the age of 30.

\begin{center}
\begin{tikzpicture}
\begin{axis}[
    ybar interval, ymin=0, ymax=14, xmin=15, xmax=40,
    xlabel={Age (years)}, ylabel={Persons ($f$)},
    xtick={15,20,25,30,35,40}, width=0.5\textwidth, height=6cm, fill=blue!60!white
]
\addplot coordinates {(15, 9) (20, 7) (25, 12) (30, 5) (35, 2) (40, 0)};
\end{axis}
\end{tikzpicture}
\end{center}

\vspace{1cm}
% ------------------------------------------
\subsection{Example 3: Age Distribution (With Class Gaps)}
\textbf{Data Provided:} Age vs. Number of persons. Classes have gaps (20-29, 30-39...), so we first calculate continuous class boundaries (19.5-29.5...).

\subsubsection*{a. Data Summarization (Using True Class Boundaries)}
\begin{table}[h]
\centering
\begin{tabular}{ccccccc}
\toprule
\textbf{Age limits} & \textbf{Boundaries} & \textbf{$f$} & \textbf{cf} & \textbf{rf} & \textbf{rf (\%)} & \textbf{Cum. (\%)} \\
\midrule
20--29 & 19.5--29.5 & 3  & 3  & 0.111 & 11.1 & 11.1 \\
30--39 & 29.5--39.5 & 10 & 13 & 0.370 & 37.0 & 48.1 \\
40--49 & 39.5--49.5 & 12 & 25 & 0.444 & 44.4 & 92.5 \\
50--59 & 49.5--59.5 & 2  & 27 & 0.074 & 7.4  & 99.9 \\
\midrule
\textbf{Total} & & \textbf{27} & & \textbf{0.999} & \textbf{99.9} & \\
\bottomrule
\end{tabular}
\end{table}

\subsubsection*{b. Graphs \& Ogive}
\begin{center}
% Histogram
\begin{tikzpicture}
\begin{axis}[
    ybar interval, ymin=0, ymax=14, xmin=19.5, xmax=59.5,
    xlabel={Age Boundaries (years)}, ylabel={Persons ($f$)},
    xtick={19.5, 29.5, 39.5, 49.5, 59.5}, width=0.45\textwidth, height=5.5cm, fill=blue!50
]
\addplot coordinates {(19.5, 3) (29.5, 10) (39.5, 12) (49.5, 2) (59.5, 0)};
\end{axis}
\end{tikzpicture}
\quad
% Ogive
\begin{tikzpicture}
\begin{axis}[
    ymin=0, ymax=30, xmin=15, xmax=65,
    xlabel={Upper Class Boundaries}, ylabel={Cumulative Freq. (cf)},
    xtick={19.5, 29.5, 39.5, 49.5, 59.5}, width=0.45\textwidth, height=5.5cm,
    title={Cumulative Frequency Curve (Ogive)}
]
\addplot[mark=square*, thick, orange] coordinates {(19.5, 0) (29.5, 3) (39.5, 13) (49.5, 25) (59.5, 27)};
\end{axis}
\end{tikzpicture}
\end{center}
\textbf{Conclusion:} The maximum number of individuals ($44.4\%$) fall within the 40-49 age group (boundaries 39.5-49.5). The frequency curve would show a left-skewed distribution as most individuals are older than 30.

\newpage
% ==========================================
% SECTION 2: QUALITATIVE DATA
% ==========================================
\section{Qualitative Data Analysis}

\subsection{Example 1: Future Profession}
\textbf{Data Provided:} Future Profession vs. No. of students ($n=30$).
\\ \textbf{a. Data Type:} This is \textbf{Qualitative Nominal Data}.

\subsubsection*{Data Table for Pie Chart}
\begin{table}[h]
\centering
\begin{tabular}{lcccc}
\toprule
\textbf{Profession} & \textbf{Freq ($f$)} & \textbf{Percentage (\%)} & \textbf{Degree ($^\circ$) calculation} & \textbf{Degree ($^\circ$)} \\
\midrule
Banker   & 12 & 40.0 & $0.400 \times 360$ & 144 \\
Teacher  & 5  & 16.7 & $0.167 \times 360$ & 60  \\
Business & 10 & 33.3 & $0.333 \times 360$ & 120 \\
Lawyer   & 3  & 10.0 & $0.100 \times 360$ & 36  \\
\midrule
\textbf{Total} & \textbf{30} & \textbf{100.0} & & \textbf{360} \\
\bottomrule
\end{tabular}
\end{table}

\subsubsection*{Pie Chart \& Bar Diagram}
\begin{center}
\begin{tikzpicture}
% Bar Chart
\begin{axis}[
    ybar,
    symbolic x coords={Banker, Teacher, Business, Lawyer},
    xtick=data,
    ymin=0, ymax=14,
    ylabel={Number of Students},
    width=0.45\textwidth, height=6cm,
    bar width=20pt,
    nodes near coords,
    fill=teal!60
]
\addplot coordinates {(Banker, 12) (Teacher, 5) (Business, 10) (Lawyer, 3)};
\end{axis}
\end{tikzpicture}
\quad
\begin{tikzpicture}
% Pie chart
\def\radius{2.5cm}
\draw[fill=blue!50] (0,0) -- (0:\radius) arc (0:144:\radius) -- cycle;
\node at (72:\radius/1.6) {Banker};

\draw[fill=red!50] (0,0) -- (144:\radius) arc (144:204:\radius) -- cycle;
\node at (174:\radius/1.4) {Teacher};

\draw[fill=green!50] (0,0) -- (204:\radius) arc (204:324:\radius) -- cycle;
\node at (264:\radius/1.6) {Business};

\draw[fill=orange!50] (0,0) -- (324:\radius) arc (324:360:\radius) -- cycle;
\node at (342:\radius/1.3) {Lawyer};
\end{tikzpicture}
\end{center}
\textbf{Comment:} The majority of students ($40\%$) want to be bankers, while the least preferred profession is Lawyer ($10\%$).

\vspace{0.5cm}
% ------------------------------------------
\subsection{Example 2: Favourite Colours (Overall \& Grouped by Gender)}
\textbf{Data Type:} Qualitative Nominal.

\subsubsection*{Grouped Bar Diagram (Male vs Female)}
\begin{table}[h]
\centering
\begin{tabular}{lcccc}
\toprule
\textbf{Colour} & \textbf{Male ($n=22$)} & \textbf{Female ($n=18$)} & \textbf{\% Male} & \textbf{\% Female} \\
\midrule
Black & 10 & 5 & 45.5\% & 27.8\% \\
Blue  & 3  & 7 & 13.6\% & 38.9\% \\
Red   & 4  & 3 & 18.2\% & 16.7\% \\
White & 5  & 3 & 22.7\% & 16.7\% \\
\bottomrule
\end{tabular}
\end{table}

\begin{center}
\begin{tikzpicture}
\begin{axis}[
    ybar,
    symbolic x coords={Black, Blue, Red, White},
    xtick=data,
    ymin=0, ymax=12,
    ylabel={Number of Students},
    width=0.7\textwidth, height=6cm,
    bar width=15pt,
    legend pos=north east,
    enlarge x limits=0.2,
]
\addplot[fill=blue!60] coordinates {(Black, 10) (Blue, 3) (Red, 4) (White, 5)};
\addplot[fill=red!60]  coordinates {(Black, 5) (Blue, 7) (Red, 3) (White, 3)};
\legend{Male, Female}
\end{axis}
\end{tikzpicture}
\end{center}
\textbf{Comment:} For males, the maximum percentage ($45.5\%$) favour Black, and the minimum ($13.6\%$) favour Blue. Conversely, for females, the maximum ($38.9\%$) favour Blue, while Red and White are tied for the minimum ($16.7\%$ each).

\newpage
% ==========================================
% SECTION 3: TIME SERIES DATA
% ==========================================
\section{Time Series Data Analysis}

\subsection{Example 1: Price of Sugar Over Time}
\textbf{a. Type of data:} This is \textbf{Time Series Data} (data collected at successive points in time). \\
\textbf{c. Write 5 examples of time series data:} 
1. Daily stock market closing prices. 
2. Annual gross domestic product (GDP) of a country. 
3. Monthly rainfall in a specific city. 
4. Hourly temperature readings. 
5. Annual sales revenue of a company.

\subsection{Example 5: Price of Sugar vs Price of Onion}
\textbf{Data Provided:}
\begin{table}[h]
\centering
\begin{tabular}{ccc}
\toprule
\textbf{Years} & \textbf{Price of Sugar (Tk)} & \textbf{Price of Onion (Tk)} \\
\midrule
2018 & 40 & 50 \\
2019 & 50 & 35 \\
2020 & 45 & 30 \\
2021 & 70 & 50 \\
2022 & 90 & 30 \\
\bottomrule
\end{tabular}
\end{table}

\subsubsection*{Line Diagram \& Interpretation}
\begin{center}
\begin{tikzpicture}
\begin{axis}[
    ymin=0, ymax=100,
    xmin=2017.5, xmax=2022.5,
    xtick={2018, 2019, 2020, 2021, 2022},
    xticklabel style={/pgf/number format/1000 sep=}, % Removes comma from years
    xlabel={Years},
    ylabel={Price (Tk)},
    width=0.8\textwidth, height=7cm,
    legend pos=north west,
    grid=major
]
\addplot[mark=*, thick, blue!70] coordinates {
    (2018, 40) (2019, 50) (2020, 45) (2021, 70) (2022, 90)
};
\addplot[mark=square*, thick, red!70] coordinates {
    (2018, 50) (2019, 35) (2020, 30) (2021, 50) (2022, 30)
};
\legend{Price of Sugar, Price of Onion}
\end{axis}
\end{tikzpicture}
\end{center}

\textbf{Comment / Conclusion:} 
The line graph clearly shows that the price of sugar has a general \textbf{increasing trend} over the years 2018 to 2022, rising steeply from 45 Tk in 2020 to 90 Tk in 2022. In contrast, the price of onions does not show a steady trend; it is highly \textbf{unstable and fluctuating}, dropping as low as 30 Tk and spiking back to 50 Tk multiple times within the same period.

\end{document}